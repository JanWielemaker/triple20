\documentclass[11pt]{article}
\usepackage{pl}
\usepackage{html}
\sloppy
\makeindex

\onefile
\htmloutput{html}				% Output directory
\htmlmainfile{index}				% Main document file
\bodycolor{white}				% Page colour

\renewcommand{\runningtitle}{SWI-Prolog/XPCE Semantic Web Library}

\newcommand{\elem}[1]{{\tt\string<#1\string>}}

\begin{document}

\title{SWI-Prolog/XPCE Semantic Web Library}
\author{Jan Wielemaker \\
	SWI, \\
	University of Amsterdam \\
	The Netherlands \\
	E-mail: \email{jan@swi.psy.uva.nl}}

\maketitle

\begin{abstract}
This document describes a library for dealing with standards from the
\url[W3C]{http://www.w3c.org/} standard for the \emph{Semantic Web}.
Like the standards themselves (RDF, RDFS and OWL) this infrastructure
is modular.  It consists of Prolog packages for reading, querying and
storing semantic web documents as well as XPCE libraries that provide
visualisation and editing.  The Prolog libraries can be used without
the XPCE GUI modules.  The library can handle upto about 2 million
\emph{RDF triples} on current commonly used hardware (256MB memory,
Pentium 1.5Ghz).
\end{abstract}

\vfill
\pagebreak
\tableofcontents

\newpage


\section{Introduction}

SWI-Prolog has started support for web-documents with the development of
a small and fast SGML/XML parser, followed by an RDF parser (early
2000). With the \file{semweb} library we provide more high level support
for manipulating semantic web documents.  The semantic web is a likely
point of orientation for knowledge representation in the future, making
a library designed in its spirit promising.


\section{Modules}

Central to this library is the module \file{rdf_db.pl}, providing
storage and basic querying for RDF triples. This triple store is filled
using the RDF parser realised by \file{rdf.pl}. The storage module can
quickly save and load (partial) databases. The modules \file{rdfs.pl}
and \file{owl.pl} add querying in terms of the more powerful RDFS and
OWL languages. Module \file{rdf_edit.pl} adds editing, undo, journaling
and change-forwarding. Finally, a variety of XPCE modules visualise and
edit the database. Figure \figref{modules} summarised the modular
design.

\postscriptfig[width=0.8\linewidth]{modules}
	{Modules for the Semantic Web library}


		 %%%%%%%%%%%%%%%%%%%%%%%%%%%%%%%%
		 %	      RDF_DB		%
		 %%%%%%%%%%%%%%%%%%%%%%%%%%%%%%%%


\section{Module rdf_db}

The central module is called \file{rdf_db}.  It provides storage and
indexed querying of RDF triples.  Triples are stored as a quintuple.
The first three elements denote the RDF triple. \arg{File} and
\arg{Line} provide information about the origin of the triple.

\begin{quote}
\{\arg{Subject} \arg{Predicate} \arg{Object} \arg{File} \arg{Line}\}
\end{quote}

The actual storage is provided by the \jargon{foreign language (C)}
module \file{rdf_db.c}.  Using a dedicated C-based implementation we
can reduced memory usage and improve indexing capabilities.%
	\footnote{The orginal implementation was in Prolog.  This
		  version was implemented in 3 hours, where the C-based
		  implementation costed a full week.  The C-based
		  implementation requires about half the memory and
		  provides about twice the performance.}
Currently the following indexing is provided.

\begin{itemize}
    \item Any of the 3 fields of the triple
    \item \arg{Subject} + \arg{Predicate} and \arg{Predicate} + \arg{Object}
    \item \arg{Predicates} are indexed on the \jargon{highest property}.  In
          other words, if predicates are related through
	  \const{subPropertyOf} predicates indexing happens on the most
	  abstract predicate. This makes calls to \index{rdf_has/4}\predref{rdf_has}{4} very
	  efficient.
    \item Literal \arg{Objects} are indexed case-insensitive to make
          case-insensitive queries fully indexed.  See \index{rdf/3}\predref{rdf}{3}.
\end{itemize}

\subsection{Query the RDF database}

\begin{description}
    \predicate{rdf}{3}{?Subject, ?Predicate, ?Object}
Elementary query for triples.  \arg{Subject} and \arg{Predicate} are
atoms representing the fully qualified URL of the resource. \arg{Object}
is either an atom representing a resource or \term{literal}{Text} if the
object is a literal value.%
	\footnote{The current implementation has no provisions for
		  XML-Schema typed literals.}
If a value of the form \infixterm{:}{NameSpaceID}{LocalName} is provided
it is expanded to a ground atom using \index{expand_goal/2}\predref{expand_goal}{2}. This implies you
can use this construct in compiled code without paying a preformance
penalty.  See also \secref{rdfns}. 
For querying purposes, \arg{Object} can be of the form
\term{literal}{+Query, -Value}, where \arg{Query} is one of
    \begin{description}
	\termitem{exact}{+Text}
    	    Perform exact, but case-insensitive match.  This query is
	    fully indexed.
	\termitem{substring}{+Text}
	    Match any literal that contains \arg{Text} as a case-insensitive
	    substring.  The query is not indexed on \arg{Object}.
	\termitem{word}{+Text}
	    Match any literal that contains \arg{Text} delimited by
	    a non alpha-numeric character, the start or end of the
	    string.  The query is not indexed on \arg{Object}.
	\termitem{prefix}{+Text}
	    Match any literal that starts with \arg{Text}.  This call
	    is intended for \jargon{completion}.  The query is not
	    indexed on \arg{Object}.
    \end{description}

    \predicate{rdf}{4}{?Subject, ?Predicate, ?Object, ?Source}
As \index{rdf/3}\predref{rdf}{3} but in addition return the source-location of the triple.  The
source is either a plain atom or a term of the format
\infixterm{:}{Atom}{Integer} where \arg{Atom} is intended to be used as
filename or URL and \arg{Integer} for representing the line-number.

    \predicate{rdf_has}{4}{?Subject, ?Predicate, ?Object, -TriplePred}
This query exploits the RDFS \const{subPropertyOf} relation.  It 
returns any triple whose stored predicate equals \arg{Predicate} or
can reach this by following the recursive \arg{subPropertyOf} relation.
The actual stored predicate is returned in \arg{TriplePred}. The example
below gets all subclasses of an RDFS (or OWL) class, even if the
relation used is not \const{rdfs:subClassOf}, but a user-defined
sub-property thereof.%
	\footnote{This predicate realises semantics defined in
		  RDF-Schema rather than RDF.  It is part of the
		  \pllib{rdf_db} module because the indexing of
		  this module incorporates the  \const{rdfs:subClassOf}
		  predicate.}

\begin{code}
subclasses(Class, SubClasses) :-
        findall(S, rdf_has(S, rdfs:subClassOf, Class), SubClasses).
\end{code}

\noindent
    \predicate{rdf_has}{3}{?Subject, ?Predicate, ?Object}
Same as \term{rdf_has}{Subject, Predicate, Object, _}.

    \predicate{rdf_subject}{1}{?Subject}
Enumerate resources appearing as a subject in a triple.  The main reason
for this predicate is to generate the known subjects \emph{without
duplicates} as one gets using \term{rdf}{Subject, _, _}.
\end{description}


\subsection{Modifying the database}		\label{sec:rdfmodify}

As depicted in \figref{modules}, there are two levels of modification.
The \file{rdf_db} module simply modifies, where the \file{rdf_edit}
library provides transactions and undo on top of this.  Applications
that wish to use the \file{rdf_edit} layer must \emph{never} use the
predicates from this section directly.

\begin{description}
    \predicate{rdf_assert}{3}{+Subject, +Predicate, +Object}
Assert a new triple into the database. This is equivalent to
\index{rdf_assert/4}\predref{rdf_assert}{4} using \arg{SourceRef} \const{user}. \arg{Subject} and
\arg{Predicate} are resources. \arg{Object} is either a resource or a
term \term{literal}{Value}. All arguments are subject to name-space
expansion (see \secref{rdfns}).

    \predicate{rdf_assert}{4}{+Subject, +Predicate, +Object, +SourceRef}
As \index{rdf_assert/3}\predref{rdf_assert}{3}, adding \arg{SourceRef} to specify the orgin of the
triple.  \arg{SourceRef} is either an atom or a term of the format
\arg{Atom}:\arg{Int} where \arg{Atom} normally refers to a filename
and \arg{Int} to the line-number where the description starts.

    \predicate{rdf_retractall}{3}{?Subject, ?Predicate, ?Object}
Removes all matching triples from the database.  Previous Prolog
implementations also provided a backtracking \predref{rdf_retract}{3},
but this proved to be rarely used and could always be replaced with
\index{rdf_retractall/3}\predref{rdf_retractall}{3}. As \index{rdf_retractall/4}\predref{rdf_retractall}{4} using an unbound \arg{SourceRef}.

    \predicate{rdf_retractall}{4}{?Subject, ?Predicate, ?Object, ?SourceRef}
As \index{rdf_retractall/4}\predref{rdf_retractall}{4}, also matching on the \arg{SourceRef}.  This is
particulary useful to update all triples coming from a loaded file.

    \predicate{rdf_update}{4}{+Subject, +Predicate, +Object, +Action}
Replaces one of the three fields on the matching triples depending
on \arg{Action}:

\begin{description}
    \termitem{subject}{Resource}
Changes the first field of the triple.
    \termitem{predicate}{Resource}
Changes the second field of the triple.
    \termitem{object}{Object}
Changes the last field of the triple to the given resource or
\term{literal}{Value}.
\end{description}
\end{description}


\subsection{Loading and saving to file}		\label{sec:rdffile}

The \file{rdf_db} module can read and write RDF-XML for import and
export as well as a binary format built for quick load and save
described in \secref{rdffastfile}.  Here are the predicates
for portable RDF load and save.

\begin{description}
    \predicate{rdf_load}{1}{+In}
Load triples from \arg{In}, which is either a stream opened for
reading or an atom specifying a filename.  This predicate calls
\index{process_rdf/3}\predref{process_rdf}{3} to read the source one description at a time, avoiding
limits to the size of the input.  If \arg{In} is a file, \index{rdf_load/1}\predref{rdf_load}{1}
provides for caching the results for quick-load using \index{rdf_load_db/1}\predref{rdf_load_db}{1}
described below.  Caching is activated by creating a directory
\file{.cache} (or \file{_cache} on Windows) in the directory holding
the \fileext{rdf} files.  Cached RDF files are loaded at approx.\
25 times the speed of RDF-XML files.

    \predicate{rdf_save}{1}{+File}
Save all known triples to the given \arg{File}.

    \predicate{rdf_save}{2}{+File, +FileRef}
Save all triples whose file-part of their \arg{SourceRef} matches
\arg{FileRef} to the given \arg{File}.   Saving arbitrary selections
is possible using predicates from \secref{partsave}.

    \predicate{rdf_source}{1}{?File}
Test or enumerate the files loaded using \index{rdf_load/1}\predref{rdf_load}{1}.

    \predicate{rdf_make}{0}{}
Re-load all RDF sourcefiles (see \index{rdf_source/1}\predref{rdf_source}{1}) that have changed since
they were loaded the last time.  This implies all triples that originate
from the file are removed and the file is re-loaded.  If the file is
cached a new cache-file is written.  Please note that the new triples
are added at the end of the database, possibly changing the order of
(conflicting) triples.
\end{description}

\subsubsection{Partial save}			\label{sec:partsave}

Sometimes it is necessary to make more arbitrary selections of material
to be saved or exchange RDF descriptions over an open network link. The
predicates in this section provide for this.

\begin{description}
    \predicate{rdf_save_header}{2}{+Stream, ?FileRef}
Save an RDF header, with the XML header, \const{DOCTYPE},
\const{ENTITY} and opening the \const{rdf:RDF} element with appropriate
namespace declarations.  It uses the primitives from \secref{rdfns} to
generate the required namespaces and desired short-name.

    \predicate{rdf_save_footer}{1}{+Stream}
Close the work opened with \index{rdf_save_header/2}\predref{rdf_save_header}{2}.

    \predicate{rdf_save_subject}{3}{+Stream, +Subject, +FileRef}
Save everything known about \arg{Subject} that matches \arg{FileRef}.
Using an variable for \arg{FileRef} saves all triples with
\arg{Subject}.
\end{description}


\subsubsection{Fast loading and saving}		\label{sec:rdffastfile}

Loading and saving RDF format is relatively slow.  For this reason we
designed a binary format that is more compact, avoids the complications
of the RDF parser and avoids repetitive lookup of (URL) identifiers.
Especially the speed improvement of about 25 times is worth-while when
loading large databases.  These predicates are used for caching by
\index{rdf_load/1}\predref{rdf_load}{1} under certain conditions.

\begin{description}
    \predicate{rdf_save_db}{1}{+File}
Save all known triples into \arg{File}.  The saved version includes the
\arg{SourceRef} information.

    \predicate{rdf_save_db}{1}{+File, +FileRef}
Save all triples with \arg{SourceRef} \arg{FileRef}, regardless of the
line-number. For example, using \const{user} all information added
using \index{rdf_assert/3}\predref{rdf_assert}{3} is stored in the database.

    \predicate{rdf_load_db}{1}{+File}
Load triples from \arg{File}.  
\end{description}


\subsection{Namespace Handling}			\label{sec:rdfns}

Prolog code often contains references to constant resources in a known
XML namespace. For example,
\const{http://www.w3.\index{org/2000}\predref{org}{2000}/01/rdf-schema\#Class} refers to the most
general notion of a class. Readability and maintability concerns require
for abstraction here.  The dynamic and multifile predicate rdf_db:\index{ns/2}\predref{ns}{2}
maintains a mapping between short meaningful names and namespace
locations very much like the XML \const{xmlns} construct.  The initial
mapping contains the namespaces required for the semantic web languages
themselves:

\begin{code}
ns(rdf,  'http://www.w3.org/1999/02/22-rdf-syntax-ns#').
ns(rdfs, 'http://www.w3.org/2000/01/rdf-schema#').
ns(owl,  'http://www.w3.org/2002/7/owl#').
ns(xsd,  'http://www.w3.org/2000/10/XMLSchema#').
ns(dc,   'http://purl.org/dc/elements/1.1/').
ns(eor,  'http://dublincore.org/2000/03/13/eor#').
\end{code}

\noindent
All predicates for the semweb libraries use \index{goal_expansion/2}\predref{goal_expansion}{2} rules to
make the SWI-Prolog compiler rewrite terms of the form
\infixterm{:}{Id}{Local} into the fully qualified URL.  In addition,
the following predicates are supplied:

\begin{description}
    \predicate{rdf_equal}{2}{Resource1, Resource2}
Defined as \infixterm{=}{Resource1, Resource2}.  As this predicate is
subject to goal-expansion it can be used to obtain or test global URL
values to readable values. The following goal unifies \arg{X} with
\const{http://www.w3.\index{org/2000}\predref{org}{2000}/01/rdf-schema\#Class} without more
runtime overhead than normal Prolog unification.

\begin{code}
        rdf_equal(rdfs:'Class', X)
\end{code}

\noindent
    \predicate{rdf_register_ns}{2}{+Alias, +URL}
Register \arg{Alias} as a shorthand for \arg{URL}.  Note that the
registration must be done before loading any files using them as
namespace aliases are handled at compiletime through \index{goal_expansion/2}\predref{goal_expansion}{2}.

    \predicate{rdf_global_id}{2}{?Alias:Local, ?Global}
Runtime translation between \arg{Alias} and \arg{Local} and a
\arg{Global} URL.  Expansion is normally done at compiletime.  This
predicate is often used to turn a global URL into a more readable
term.

    \predicate{rdf_global_term}{2}{+Term0, -Term}
Expands all \arg{Alias}:\arg{Local} in \arg{Term0} and return the
result in \arg{Term}.  Use infrequently for runtime expansion of
namespace identifiers.

    \predicate{rdf_split_url}{3}{?Base, ?Local, ?URL}
Split a URL into a prefix and local part if used in mode -,-,+
or simply behave as \index{atom_concat/3}\predref{atom_concat}{3} in other modes.  The \arg{URL}
is split on the last \chr{\#} or \chr{/} character.
\end{description}


\subsection{Miscellaneous predicates}

This section describes the remaining predicates of the \file{rdf_db}
module.

\begin{description}
    \predicate{rdf_node}{1}{-Id}
Generate a unique reference.  The returned atom is guaranteed not to
occur in the current database in any field of any triple.

    \predicate{rdf_source_location}{2}{+Subject, -SourceRef}
Return the source-location as \arg{File}:\arg{Line} of the first triple
that is about \arg{Subject}.

    \predicate{rdf_statistics}{1}{?Statistics}
Report statistics collected by the \file{rdf_db} module.  Defined
values for \arg{Statistics} are:

    \begin{description}
	\termitem{lookup}{?Index, -Count}
Number of lookups using a pattern of instantiated fields.  \arg{Index}
is a term \term{rdf}{S,P,O}, where \arg{S}, \arg{P} and \arg{O} are
either \const{+} or \const{-}.  For example \term{rdf}{+,+,-} returns
the lookups with subject and predicate specified and object unbound.

	\termitem{properties}{-Count}
Number of unique values for the second field of the triple set.

	\termitem{sources}{-Count}
Number of files loaded through \index{rdf_load/1}\predref{rdf_load}{1}.

	\termitem{subjects}{-Count}
Number of unique values for the first field of the triple set.

	\termitem{triples}{-Count}
Total number of triples in the database.
        \end{description}

    \predicate{rdf_match_label}{3}{+Method, +Search, +Atom}
True if \arg{Search} matches \arg{Atom} as defined by \arg{Method}.
All matching is performed case-insensitive.  Defines methods are:
    \begin{description}
	\termitem{exact}{}
    	    Perform exact, but case-insensitive match.
	\termitem{substring}{}
	    \arg{Search} is a sub-string of \arg{Text}.
	\termitem{word}{}
	    \arg{Search} appears as a whole-word in \arg{Text}.
	\termitem{prefix}{}
	    \arg{Text} start with \arg{Search}.
    \end{description}
\end{description}


\subsection{Issues with rdf_db}				\label{sec:rdfissues}

This RDF low-level module has been created after two year experimenting
with a plain Prolog based module and a brief evaluation of a second
generation pure Prolog implementation. The was to be able to handle upto
aboud 2 million triples on standard (notebook) hardware and deal
efficiently with \const{subPropertyOf} which was identified as a
crucial feature of RDFS to realise fusion of different data-sets.

The following issues are identified and not solved in suitable manner.

\begin{description}
    \item [Logical update] as provided by Prolog means that active queries
are not affected by subsequent modification of the database. The current
C-based implementation adheres the \jargon{immediate} update model,
mainly because the current foreign language interface does not provide
the required information to realise logical updates in C.

    \item [Loops on \const{subPropertyOf}] may cause incorrect answers
as well as infinite loops.  This must be fixed.  The design already
accomodates for this, but the implementation is incomplete.

    \item [\const{subPropertyOf} of \const{subPropertyOf}] is not
supported.
\end{description}


		 %%%%%%%%%%%%%%%%%%%%%%%%%%%%%%%%
		 %	       RDFS		%
		 %%%%%%%%%%%%%%%%%%%%%%%%%%%%%%%%

\section{Module rdfs}				\label{sec:rdfs}

\index{RDF-Schema}%
The \pllib{rdfs} library adds interpretation of the triple store in
terms of concepts from RDF-Schema (RDFS).  

\subsection{Hierarchy and class-individual relations}

The predicates in this section explore the \const{rdfs:subPropertyOf},
\const{rdfs:subClassOf} and \const{rdf:type} relations.  Note that the
most fundamental of these, \const{rdfs:subPropertyOf}, is also used 
by \index{rdf_has/[3,4]}\predref{rdf_has}{[3,4]}.

\begin{description}
    \predicate{rdfs_subproperty_of}{2}{?SubProperty, ?Property}
True if \arg{SubProperty} is equal to \arg{Property} or \arg{Property}
can be reached from \arg{SubProperty} following the
\const{rdfs:subPropertyOf} relation. It can be used to test as well as
generate sub-properties or super-properties. Note that the commonly used
semantics of this predicate is wired into \index{rdf_has/[3,4]}\predref{rdf_has}{[3,4]}.%
	\bug{The current implementation cannot deal with
	     cycles}.%
	\bug{The current implementation cannot deal with predicates
	     that are an \const{rdfs:subPropertyOf} of
	     \const{rdfs:subPropertyOf}, such as
	     \const{owl:samePropertyAs}.}

    \predicate{rdfs_subclass_of}{2}{?SubClass, ?Class}
True if \arg{SubClass} is equal to \arg{Class} or \arg{Class}
can be reached from \arg{SubClass} following the
\const{rdfs:subClassOf} relation. It can be used to test as
well as generate sub-classes or super-classes.%
	\bug{The current implementation cannot deal with
	     cycles}.

    \predicate{rdfs_class_property}{2}{+Class, ?Property}
True if the domain of \arg{Property} includes \arg{Class}.  Used to
generate all properties that apply to a class.

    \predicate{rdfs_individual_of}{2}{?Resource, ?Class}
True if \arg{Resource} is an indivisual of \arg{Class}.  This implies
\arg{Resource} has an \const{rdf:type} property that refers to
\arg{Class} or a sub-class thereof.  Can be used to test, generate
classes \arg{Resource} belongs to or generate individuals described
by \arg{Class}.
\end{description}

\subsection{Collections and Containers}

\index{parseType,Collection}%
\index{Collection,parseType}%
The RDF construct \const{rdf:parseType}=\const{Collection} constructs
a list using the \const{rdf:first} and \const{rdf:next} relations.

\begin{description}
    \predicate{rdfs_member}{2}{?Resource, +Set}
Test or generate the members of \arg{Set}.  \arg{Set} is either an
individual of \const{rdf:List} or \const{rdf:Container}.

    \predicate{rdfs_list_to_prolog_list}{2}{+Set, -List}
Convert \arg{Set}, which must be an individual of \const{rdf:List} into
a Prolog list of objects.
\end{description}

\subsection{Labels and textual search}

Textual search is partly handled by the predicates from the
\pllib{rdf_db} module and its underlying C-library.  For example,
literal objects are hashed case-insensitive to speed up the commonly
used case-insensitive search.

\begin{description}
    \predicate{rdfs_label}{2}{?Resource, ?Label}
Extract the label from \arg{Resource} or generate all resources with
the given \arg{Label}. The label is either associated using
a sub-property of \const{rdfs:label} or it is extracted from the URL
using \index{rdf_split_url/3}\predref{rdf_split_url}{3}.

    \predicate{rdfs_ns_label}{2}{?Resource, ?Label}
Similar to \index{rdfs_label/2}\predref{rdfs_label}{2}, but prefixes the result using the declared
namespace alias (see \secref{rdfns}) to facilitate user-friendly labels
in applications using multiple namespaces that may lead to confusion.

    \predicate{rdfs_find}{5}{+String, +Description, +Properties, +Method, -Subject}
\index{search}%
Find (on backtracking) \arg{Subject}s that satisfy a search
specification for textual attributes.  \arg{String} is the string
searched for. \arg{Description} is an OWL description (see \secref{owl})
specifying candidate resources. \arg{Properties} is a list of properties
to search for literal objects where \arg{rdfs:label} is replaced by a
call to \index{rdfs_label/2}\predref{rdfs_label}{2} and finally, \arg{Method} defines the textual
matching algorithm.  All textual mapping is performed case-insensitive.
The matching-methods are described with \index{rdf_match_label/3}\predref{rdf_match_label}{3}.
\end{description}


		 %%%%%%%%%%%%%%%%%%%%%%%%%%%%%%%%
		 %	       OWL		%
		 %%%%%%%%%%%%%%%%%%%%%%%%%%%%%%%%

\section{Module owl}				\label{sec:owl}

\index{OWL}%
This module provides a \emph{very} limited interpretation to some
OWL concepts used in the current visualisation tools.  Its main task
is to describe an evaluate restrictions on properties. OWL is
interpreted as OWL-Full, the OWL dialect that is a superset of
RDF-Schema.

\index{Description, OWL}%
Central to this library is the Prolog representation for a
\jargon{Description}.  Predicates from this library produce these
descriptions, \index{owl_satisfies/2}\predref{owl_satisfies}{2} tests whether a resource satisfies
a description.  The Prolog description is a nested term with the
following syntax.












		 %%%%%%%%%%%%%%%%%%%%%%%%%%%%%%%%
		 %	     RDF-EDIT		%
		 %%%%%%%%%%%%%%%%%%%%%%%%%%%%%%%%

\section{Module rdf_edit}			\label{sec:rdfedit}

\index{undo}\index{journal}\index{transactions}%
The module \file{rdf_edit.pl} is a layer than encasulates the
modification predicates from \secref{rdfmodify} for use from
a (graphical) editor of the triple store.  It adds the
following features:

\begin{itemlist}
    \item [Transaction management]
Modifications are grouped into \emph{transactions} to safeguard
the system from failing operations as well as provide meaningfull
chunks for undo and journalling.

    \item [Undo]
Undo and redo-transactions using a single mechanism to support
user-friendly editing.

    \item [Journalling]
Record all actions to support analysis, versioning, crash-recovery
and an alternative to saving.
\end{itemlist}

\subsection{Transaction management}

Transactions group low-level modification actions together.

\begin{description}
    \predicate{rdfe_transaction}{1}{:Goal}
Run \arg{Goal}, recording all modifications to the triple store made
through \secref{rdfeencap}.  Execution is performed as in \index{once/1}\predref{once}{1}.  If
\arg{Goal} succeeds the changes are committed.  If \arg{Goal} fails 
or throws an exception the changes are reverted.

Transactions may be nested.  A failing nested transaction only reverts
the actions performed inside the nested transaction.  If the outer
transaction succeeds it is committed normally.  Contrary, if the
outer transaction fails, comitted nested transactions are reverted
as well.

A successful outer transaction (`level-0') may be undone using
\index{rdfe_undo/0}\predref{rdfe_undo}{0}.
\end{description}


\subsection{Encapsulated predicates}		\label{sec:rdfeencap}

The following predicates encapsulate predicates from the \file{rdf_db}
module that modify the triple store. These predicates can only be called
when inside a \emph{transaction}.  See \index{rdfe_transaction/1}\predref{rdfe_transaction}{1}.

\begin{description}
    \predicate{rdfe_assert}{3}{+Subject, +Predicate, +Object}
Encapsulates \index{rdf_assert/3}\predref{rdf_assert}{3}.
    \predicate{rdfe_retractall}{3}{?Subject, ?Predicate, ?Object}
Encapsulates \index{rdf_retractall/3}\predref{rdf_retractall}{3}.
    \predicate{rdfe_update}{4}{+Subject, +Predicate, +Object, +Action}
Encapsulates \index{rdf_update/4}\predref{rdf_update}{4}.
    \predicate{rdf_load}{1}{+In}
Encapsulates \index{rdf_load/1}\predref{rdf_load}{1}.
\end{description}

\subsection{High-level modification predicates}	\label{sec:rdfeedit}

This section describes a (yet very incomplete) set of more high-level
operations one would like to be able to perform.  Eventually this set
may include operations based on RDFS and OWL.

\begin{description}
    \predicate{rdfe_delete}{1}{+Resource}
Delete all traces of \arg{resource}.  This implies all triples where
\arg{Resource} appears as \emph{subject}, \emph{predicate} or
\emph{object}.  This predicate starts a transation.
\end{description}

\subsection{Undo}

\index{undo}%
Undo aims at user-level undo operations from a (graphical) editor.

\begin{description}
    \predicate{rdfe_undo}{0}{}
Revert the last outermost (`level 0') transaction (see
\index{rdfe_transaction/1}\predref{rdfe_transaction}{1}). Successive calls go further back in history. Fails
if there is no more undo information.

    \predicate{rdfe_redo}{0}{}
Revert the last \index{rdfe_undo/0}\predref{rdfe_undo}{0}.  Successive calls revert more \index{rdfe_undo/0}\predref{rdfe_undo}{0}
operations.   Fails if there is no more redo information.

    \predicate{rdfe_can_undo}{0}{}
Test if there is another transaction that can be reverted.  Used for
activating menus in a graphical environment.

    \predicate{rdfe_can_redo}{0}{}
Test if there is another undo that can be reverted.  Used for
activating menus in a graphical environment.
\end{description}

\subsection{Journalling}

\index{journal}%
Optionally, every action through this module is immediately send to a
\jargon{journal-file}. The journal provides a full log of all actions
with a time-stamp that may be used for inspection of behaviour, version
management, crash-recovery or an alternative to regular save operations.

\begin{description}
    \predicate{rdfe_open_journal}{2}{+File, +Mode}
Open a existing or new journal.  If \arg{Mode} equala \const{append}
and \arg{File} exists, the journal is first replayed. See
\index{rdfe_replay_journal/1}\predref{rdfe_replay_journal}{1}.  If \arg{Mode} is \const{write} the journal is
truncated if it exists.

    \predicate{rdfe_close_journal}{0}{}
Close the currently open journal.

    \predicate{rdfe_current_journal}{1}{-Path}
Test whether there is a journal and to which file the actions are
journalled.

    \predicate{rdfe_replay_journal}{1}{+File}
Read a jorunal, replaying all actions in it.  To do so, the system
reads the journal a transaction at a time.  If the transaction is
closed with a \emph{commit} it executes the actions inside the journal.
If it is closed with a \emph{rollback} or not closed at all due to a
crash the actions inside the journal are discarded.  Using this
predicate only makes sense to inspect the state at the end of a journal
without modifying the journal.  Normally a journal is replayed using the
\const{append} mode of \index{rdfe_open_journal/2}\predref{rdfe_open_journal}{2}.
\end{description}


\subsection{Broadcasting change events}

\index{event}\index{broadcast}%
To realise a modular graphical interface for editing the triple store,
the system must use some sort of \emph{event} mechanism. This is
implemented by the XPCE library \pllib{broadcast} which is described
in the \url[XPCE User
Guide]{http://www.swi.psy.uva.nl/projects/xpce/UserGuide/libbroadcast.html}.
In this section we describe the terms brodcasted by the library.

\begin{description}
    \termitem{rdf_transaction}{+Id}
A `level-0' transaction has been committed. The system passes the
identifier of the transaction in \arg{Id}. In the current implementation
there is no way to find out what happened inside the transaction.  This
is likely to change in time.

If a transaction is reverted due to failure or exception \emph{no} event
is broadcasted.  The initiating GUI element is supposed to handle this
possibility itself and other components are not affected as the triple
store is not changed.

    \termitem{rdf_undo}{+Type, +Id}
This event is broadcasted after an \index{rdfe_undo/0}\predref{rdfe_undo}{0} or \index{rdfe_redo/0}\predref{rdfe_redo}{0}.
\arg{Type} is one of \const{undo} or \const{redo} and \arg{Id}
identifies the transaction as above.
\end{description}


		 %%%%%%%%%%%%%%%%%%%%%%%%%%%%%%%%
		 %	      FOOTER		%
		 %%%%%%%%%%%%%%%%%%%%%%%%%%%%%%%%

\printindex

\end{document}


